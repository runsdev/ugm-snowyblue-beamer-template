\documentclass[aspectratio=169,professionalfonts]{beamer}

% ----- THEME AND PACKAGES -----
\usetheme{UGM} 
\usefonttheme{professionalfonts}
\usepackage[type1, sfdefault]{atkinson}
\usepackage[T1]{fontenc}

% Essential packages for enhanced features
\usepackage{amsmath}
\usepackage{booktabs}        % For better tables
\usepackage{multicol}        % For multi-column layouts
\usepackage{tcolorbox}       % For colored boxes
\usepackage{listings}        % For code listings
\usepackage{pgfplots}        % For plots and graphs
\usepackage{algorithm2e}     % For algorithms
\usepackage{fontawesome5}    % For icons
\usepackage{tikz}
\usetikzlibrary{mindmap,trees,arrows,shapes,positioning}
\usepackage{media9}          % For embedded videos
\usepackage[table]{xcolor}

% Set up code listing style
\lstdefinestyle{mystyle}{
  backgroundcolor=\color{backcolour},   
  commentstyle=\color{codegreen},
  keywordstyle=\color{ugmBlue},
  numberstyle=\tiny\color{codegray},
  stringstyle=\color{codepurple},
  basicstyle=\ttfamily\footnotesize,
  breakatwhitespace=false,         
  breaklines=true,                 
  captionpos=b,                    
  keepspaces=true,                 
  numbers=left,                    
  numbersep=5pt,                  
  showspaces=false,                
  showstringspaces=false,
  showtabs=false,                  
  tabsize=2
}
\lstset{style=mystyle}

% Define colors for highlighting
\definecolor{codegreen}{rgb}{0,0.6,0}
\definecolor{codegray}{rgb}{0.5,0.5,0.5}
\definecolor{codepurple}{rgb}{0.58,0,0.82}
\definecolor{backcolour}{rgb}{0.95,0.95,0.92}

% ----- PRESENTATION METADATA -----
\title{UGM Beamer \\Template Guidelines}
\subtitle{\textbf{UGM Beamer Theme} $|$ Template Documentation}
\author{Harun $|$ {\href{https://github.com/runsdev}{github.com/runsdev}}}
\date{\today}
\logo{ugm_logo.png}

\begin{document}

% ----- TITLE FRAME -----
\begin{frame}[plain]
    \titlepage
\end{frame}

% ----- TABLE OF CONTENTS -----
\begin{frame}
    \frametitle{Contents}
    
    \begin{multicols}{2}
        \tableofcontents
    \end{multicols}
\end{frame}

% ----- SECTION: BASIC SLIDES -----
\section{Basic Slide Elements}

\begin{frame}
    \frametitle{Basic Text Formatting}
    
    \begin{itemize}
        \item Normal text with \textbf{bold}, \textit{italic}, and \textcolor{ugmBlue}{colored text}
        \item Bullet points with standard indentation
        \item \alert{Highlighted text} for emphasis
    \end{itemize}
    
    \begin{enumerate}
        \item Numbered lists are created with \texttt{enumerate}
        \item Second item in the list
        \item Third item with \textbf{formatting}
    \end{enumerate}
\end{frame}

\begin{frame}
    \frametitle{Basic Text Formatting}
    
    \begin{block}{Standard Block}
        This is a standard block environment for highlighting content.
    \end{block}
    
    \begin{alertblock}{Alert Block}
        This block is used for warnings or important notes.
    \end{alertblock}
    
    \begin{exampleblock}{Example Block}
        This block is used for examples.
    \end{exampleblock}
\end{frame}

\begin{frame}
    \frametitle{Basic Math Formatting}
    
    Inline math: $E = mc^2$ is Einstein's famous equation.
    
    Display math:
    \[
    \nabla \times \vec{E} = -\frac{\partial \vec{B}}{\partial t}
    \]
    
    Equation with numbering:
    \begin{equation}
        \nabla \times \vec{B} = \mu_0 \vec{J} + \mu_0 \varepsilon_0 \frac{\partial \vec{E}}{\partial t}
    \end{equation}
    
    Align environment for multiple equations:
    \begin{align}
        E &= mc^2 \\
        m &= \frac{E}{c^2}
    \end{align}
\end{frame}

% ----- SECTION: LAYOUTS -----
\section{Multi-Column Layouts}

\begin{frame}
    \frametitle{Two-Column Layout with Beamer Columns}
    
    \begin{columns}[T]
        \begin{column}{0.48\textwidth}
            \textbf{Left Column}
            \begin{itemize}
                \item First bullet point
                \item Second bullet point
                \item Third bullet point with longer text that may wrap to the next line
            \end{itemize}
            
            Some regular text in the left column.
        \end{column}
        
        \begin{column}{0.48\textwidth}
            \textbf{Right Column}
            
            \includegraphics[width=\textwidth]{example-image-a}
            \centerline{\small Image Caption}
        \end{column}
    \end{columns}
\end{frame}

\begin{frame}
    \frametitle{Three-Column Layout}
    
    \begin{columns}[T]
        \begin{column}{0.3\textwidth}
            \textbf{Column 1}
            \begin{itemize}
                \item Item 1
                \item Item 2
            \end{itemize}
        \end{column}
        
        \begin{column}{0.3\textwidth}
            \textbf{Column 2}
            \begin{itemize}
                \item Item A
                \item Item B
            \end{itemize}
        \end{column}
        
        \begin{column}{0.3\textwidth}
            \textbf{Column 3}
            \begin{itemize}
                \item Item X
                \item Item Y
            \end{itemize}
        \end{column}
    \end{columns}
\end{frame}

\begin{frame}
    \frametitle{Uneven Column Layout}
    
    \begin{columns}[T]
        \begin{column}{0.65\textwidth}
            \textbf{Wider Column}
            
            This column contains more content and takes up 65\% of the slide width.
            
            \begin{itemize}
                \item The width can be adjusted as needed
                \item The layout is flexible
                \item Content will flow within the specified width
            \end{itemize}
        \end{column}
        
        \begin{column}{0.3\textwidth}
            \textbf{Narrower Column}
            
            This column is only 30\% wide.
            
            \includegraphics[width=\textwidth]{example-image-b}
        \end{column}
    \end{columns}
\end{frame}

% ----- SECTION: TABLES -----
\section{Tables}

\begin{frame}
    \frametitle{Basic Table}
    
    \begin{table}
        \centering
        \begin{tabular}{|l|c|r|}
            \hline
            \textbf{Left} & \textbf{Center} & \textbf{Right} \\
            \hline
            Data 1 & 123 & 45.67 \\
            Data 2 & 456 & 89.01 \\
            Data 3 & 789 & 23.45 \\
            \hline
        \end{tabular}
        \caption{Basic table with borders}
    \end{table}
    
    \vspace{1em}
    
    \begin{tcolorbox}[colback=ugmLightGrey,colframe=ugmBlue,title=Table Usage Tip]
        Use the \texttt{|} character to create vertical lines and \texttt{\textbackslash hline} for horizontal lines.
    \end{tcolorbox}
\end{frame}

\begin{frame}
    \frametitle{Professional Table with Booktabs}
    
    \begin{table}
        \centering
        \begin{tabular}{lrr}
            \toprule
            \textbf{Method} & \textbf{Accuracy (\%)} & \textbf{Time (s)} \\
            \midrule
            Method A & 95.2 & 1.23 \\
            Method B & 97.8 & 2.56 \\
            Method C & 98.1 & 4.78 \\
            \bottomrule
        \end{tabular}
        \caption{Professional table with booktabs package}
    \end{table}
    
    \begin{tcolorbox}[colback=ugmLightGrey,colframe=ugmBlue,title=Booktabs Tip]
        Use \texttt{\textbackslash toprule}, \texttt{\textbackslash midrule}, and \texttt{\textbackslash bottomrule} for professional tables.
    \end{tcolorbox}
\end{frame}

\begin{frame}
    \frametitle{Colored Table}
    
    \begin{table}
        \centering
        \rowcolors{2}{ugmLightGrey}{white}
        \begin{tabular}{lcr}
            \rowcolor{ugmBlue!40}
            \textcolor{white}{\textbf{Category}} & \textcolor{white}{\textbf{Value}} & \textcolor{white}{\textbf{Percentage}} \\
            Category A & 45.2 & 22\% \\
            Category B & 32.1 & 16\% \\
            Category C & 78.9 & 39\% \\
            Category D & 47.3 & 23\% \\
        \end{tabular}
        \caption{Table with alternating row colors}
    \end{table}
    
    \begin{tcolorbox}[colback=ugmLightGrey,colframe=ugmBlue,title=Color Tip]
        Use \texttt{\textbackslash rowcolors\{2\}\{color1\}\{color2\}} to alternate row colors starting from row 2.
    \end{tcolorbox}
\end{frame}

% ----- SECTION: IMAGES AND FIGURES -----
\section{Images and Figures}

\begin{frame}
    \frametitle{Basic Image Inclusion}
    
    \begin{columns}[T]
        \begin{column}{0.48\textwidth}
            \includegraphics[width=\textwidth]{example-image-a}
            \centerline{\small Full-width image with caption}
            
            \vspace{1em}
            
            % \begin{tcolorbox}[colback=ugmLightGrey,colframe=ugmBlue,title=Image Tip]
            %     Use \texttt{width=\textbackslash textwidth} to make the image fill the column width.
            % \end{tcolorbox}
        \end{column}
        
        \begin{column}{0.48\textwidth}
            \begin{figure}
                \includegraphics[width=0.8\textwidth]{example-image-b}
                \caption{Image using figure environment}
            \end{figure}
            
            % \begin{itemize}
            %     \item Use \texttt{figure} environment for formal figures
            %     \item Use \texttt{centerline} for simple captions
            %     \item Control size with \texttt{width} parameter
            % \end{itemize}
        \end{column}
    \end{columns}
\end{frame}

\begin{frame}
    \frametitle{Basic Image Inclusion}
    
    \begin{columns}[T]
        \begin{column}{0.48\textwidth}   
            \begin{tcolorbox}[colback=ugmLightGrey,colframe=ugmBlue,title=Image Tip]
                Use \texttt{width=\textbackslash textwidth} to make the image fill the column width.
            \end{tcolorbox}
        \end{column}
        
        \begin{column}{0.48\textwidth}
            \begin{itemize}
                \item Use \texttt{figure} environment for formal figures
                \item Use \texttt{centerline} for simple captions
                \item Control size with \texttt{width} parameter
            \end{itemize}
        \end{column}
    \end{columns}
\end{frame}

\begin{frame}
    \frametitle{Image Grid Layout}
    
    \begin{center}
        \begin{tabular}{cc}
            \includegraphics[width=0.1\textwidth]{example-image-a} &
            \includegraphics[width=0.1\textwidth]{example-image-b} \\
            \small Image A & \small Image B \\[1em]
            \includegraphics[width=0.1\textwidth]{example-image-c} &
            \includegraphics[width=0.1\textwidth]{example-image} \\
            \small Image C & \small Image D \\
        \end{tabular}
    \end{center}
    
    \begin{tcolorbox}[colback=ugmLightGrey,colframe=ugmBlue,title=Grid Tip]
        Use a \texttt{tabular} environment to create a grid of images with captions.
    \end{tcolorbox}
\end{frame}

\begin{frame}
    \frametitle{Including Videos}
    
    \begin{center}
        \includemedia[
            width=0.8\textwidth,
            height=0.6\textwidth,
            activate=pageopen,
            flashvars={
                modestbranding=1 % Hide YouTube logo
            }
        ]{}{example-movie.mp4}
    \end{center}
    
    \begin{tcolorbox}[colback=ugmLightGrey,colframe=ugmBlue,title=Video Tip]
        \begin{itemize}
            \item Use the \texttt{media9} package for embedded videos
            \item Videos will play in PDF viewers that support multimedia (e.g., Adobe Reader)
            \item Parameters can control autoplay, controls, etc.
        \end{itemize}
    \end{tcolorbox}
\end{frame}

% ----- SECTION: DIAGRAMS AND CHARTS -----
\section{Diagrams and Charts}

\begin{frame}
    \frametitle{TikZ Diagrams}
    
    \begin{columns}[T]
        \begin{column}{0.48\textwidth}
            \begin{tikzpicture}[scale=0.7, transform shape]
                % Nodes
                \node[draw, fill=ugmLightGrey, circle, minimum size=1cm] (A) at (0,0) {A};
                \node[draw, fill=ugmLightGrey, circle, minimum size=1cm] (B) at (3,1) {B};
                \node[draw, fill=ugmLightGrey, circle, minimum size=1cm] (C) at (3,-1) {C};
                \node[draw, fill=ugmLightGrey, circle, minimum size=1cm] (D) at (6,0) {D};
                
                % Arrows
                \draw[->, thick] (A) -- (B);
                \draw[->, thick] (A) -- (C);
                \draw[->, thick] (B) -- (D);
                \draw[->, thick] (C) -- (D);
            \end{tikzpicture}
            \centerline{\small Simple network diagram}
        \end{column}
        
        \begin{column}{0.48\textwidth}
            \begin{tikzpicture}[scale=0.7, transform shape]
                % Flowchart
                \node[draw, fill=ugmLightGrey, rectangle, rounded corners, minimum width=2cm] (start) at (0,0) {Start};
                \node[draw, fill=ugmLightGrey, rectangle, rounded corners, minimum width=2cm] (process) at (0,-2) {Process};
                \node[draw, fill=ugmLightGrey, diamond, aspect=2, minimum width=2cm] (decision) at (0,-4) {Decision};
                \node[draw, fill=ugmLightGrey, rectangle, rounded corners, minimum width=2cm] (end) at (0,-6) {End};
                
                % Arrows
                \draw[->, thick] (start) -- (process);
                \draw[->, thick] (process) -- (decision);
                \draw[->, thick] (decision) -- node[right] {Yes} (end);
                \draw[->, thick] (decision) -- ++(2,0) |- (process);
                \node at (2,-3) {No};
            \end{tikzpicture}
            \centerline{\small Simple flowchart}
        \end{column}
    \end{columns}
\end{frame}

\begin{frame}
    \frametitle{Graphs with PGFPlots}
    
    \begin{columns}[T]
        \begin{column}{0.48\textwidth}
            \begin{tikzpicture}
                \begin{axis}[
                    width=\textwidth,
                    height=5cm,
                    xlabel={x},
                    ylabel={y},
                    title={Simple Line Plot},
                    legend pos=north west
                ]
                    \addplot[color=ugmBlue,mark=*] coordinates {
                        (0,0) (1,1) (2,4) (3,9) (4,16)
                    };
                    \addplot[color=red,mark=square] coordinates {
                        (0,0) (1,2) (2,4) (3,6) (4,8)
                    };
                    \legend{$x^2$, $2x$}
                \end{axis}
            \end{tikzpicture}
        \end{column}
        
        \begin{column}{0.48\textwidth}
            \begin{tikzpicture}
                \begin{axis}[
                    width=\textwidth,
                    height=5cm,
                    xlabel={Category},
                    ylabel={Value},
                    title={Bar Chart},
                    ybar,
                    symbolic x coords={A, B, C, D},
                    xtick=data
                ]
                    \addplot[fill=ugmBlue] coordinates {
                        (A, 10) (B, 7) (C, 12) (D, 5)
                    };
                \end{axis}
            \end{tikzpicture}
        \end{column}
    \end{columns}
    
    Use the \texttt{pgfplots} package for creating professional charts and graphs.
\end{frame}

\begin{frame}
    \frametitle{More Chart Examples}
    
    \begin{columns}[T]
        \begin{column}{0.48\textwidth}
            \begin{tikzpicture}
                \begin{axis}[
                    width=\textwidth,
                    height=5cm,
                    xlabel={x},
                    ylabel={Probability},
                    title={Normal Distribution},
                    domain=-3:3,
                    samples=100
                ]
                    \addplot[color=ugmBlue,thick] {exp(-x^2/2)/sqrt(2*pi)};
                \end{axis}
            \end{tikzpicture}
        \end{column}
        
        \begin{column}{0.48\textwidth}
            \begin{tikzpicture}
                \begin{axis}[
                    width=\textwidth,
                    height=5cm,
                    title={Pie Chart},
                    view={0}{90},
                    colormap={whiteblue}{color(0cm)=(ugmBlue); color(1cm)=(ugmLightGrey)},
                    colorbar,
                    colorbar style={title=Value}
                ]
                    \addplot3[
                        surf,
                        shader=interp,
                        domain=0:1,
                        domain y=0:360
                    ] 
                    ({y*cos(x)}, {y*sin(x)}, {y^2});
                \end{axis}
            \end{tikzpicture}
        \end{column}
    \end{columns}
\end{frame}

% ----- SECTION: CODE LISTINGS -----
\section{Code Listings}

\begin{frame}[fragile]
    \frametitle{Basic Code Listing}
    
    \begin{lstlisting}[language=C++, caption=Hello World Example]
#include <iostream>

int main() {
    // This is a comment
    std::cout << "Hello, World!" << std::endl;
    return 0;
}
    \end{lstlisting}
    
    \begin{tcolorbox}[colback=ugmLightGrey,colframe=ugmBlue,title=Listing Tip]
        \begin{itemize}
            \item Specify the language for syntax highlighting
            \item Use the \texttt{listings} and \texttt{[fragile]} for frames with listings
        \end{itemize}
    \end{tcolorbox}
\end{frame}

\begin{frame}[fragile]
    \frametitle{Code Listing with Line Numbers and Highlighting}
    
    \begin{lstlisting}[language=Python, 
                      caption=Python Example, 
                      numbers=left,
                      numberstyle=\tiny\color{codegray},
                      keywordstyle=\color{ugmBlue},
                      commentstyle=\color{codegreen},
                      stringstyle=\color{codepurple},
                      basicstyle=\ttfamily\footnotesize,
                      breaklines=true]
def calculate_factorial(n):
    """
    Calculate the factorial of a number using recursion.
    """
    if n <= 1:
        return 1  # Base case
    else:
        return n * calculate_factorial(n-1)  # Recursive case

# Test the function
for i in range(5):
    print(f"Factorial of {i} is {calculate_factorial(i)}")
    \end{lstlisting}
\end{frame}

\begin{frame}[fragile]
    \frametitle{Inline Code and Algorithms}
    
    \begin{columns}[T]
        \begin{column}{0.48\textwidth}
            Inline code: \lstinline|int x = 42;|
            
            \begin{lstlisting}[language=SQL, caption=SQL Query]
SELECT 
    students.name,
    AVG(grades.score) as avg_score
FROM students
JOIN grades ON students.id = grades.student_id
GROUP BY students.id
HAVING avg_score > 80
ORDER BY avg_score DESC;
            \end{lstlisting}
        \end{column}
        
        \begin{column}{0.48\textwidth}
            \begin{algorithm}[H]
                \SetAlgoLined
                \KwIn{Array $A$ of size $n$}
                \KwOut{Sorted array $A$}
                \For{$i \gets 1$ \KwTo $n-1$}{
                    $key \gets A[i]$\;
                    $j \gets i-1$\;
                    \While{$j \geq 0$ and $A[j] > key$}{
                        $A[j+1] \gets A[j]$\;
                        $j \gets j-1$\;
                    }
                    $A[j+1] \gets key$\;
                }
                \caption{Insertion Sort}
            \end{algorithm}
        \end{column}
    \end{columns}
\end{frame}

% ----- SECTION: CUSTOM BOXES -----
\section{Custom Boxes and Decorations}

\begin{frame}
    \frametitle{TColor Boxes}
    
    \begin{tcolorbox}[
        colback=ugmLightGrey,
        colframe=ugmBlue,
        title=Standard Box,
        fonttitle=\bfseries
    ]
        This is a standard colored box with a title.
        
        \begin{itemize}
            \item You can include lists
            \item And other content inside
        \end{itemize}
    \end{tcolorbox}
    
    \begin{tcolorbox}[
        colback=red!5,
        colframe=red!80!black,
        title=Warning Box,
        fonttitle=\bfseries
    ]
        This box uses different colors to indicate warnings or critical information.
    \end{tcolorbox}
    
    \begin{tcolorbox}[
        enhanced,
        colback=ugmLightGrey,
        colframe=ugmBlue,
        title=Box with Shadow,
        fonttitle=\bfseries,
        drop shadow
    ]
        This box has a drop shadow effect.
    \end{tcolorbox}
\end{frame}

\begin{frame}
    \frametitle{Custom TColor Boxes}
    
    \begin{tcolorbox}[
        enhanced,
        colback=ugmLightGrey,
        colframe=ugmBlue,
        title=Note Box,
        fonttitle=\bfseries,
        overlay unbroken and first={
            \node[anchor=west, inner sep=0pt] 
            at ([xshift=10pt]frame.north west) 
            {\faInfoCircle\ };
        }
    ]
        Important note with an icon in the title.
    \end{tcolorbox}
    
    \begin{tcolorbox}[
        enhanced,
        colback=yellow!10,
        colframe=yellow!50!black,
        title=Tip Box,
        fonttitle=\bfseries,
        overlay unbroken and first={
            \node[anchor=west, inner sep=0pt] 
            at ([xshift=10pt]frame.north west) 
            {\faLightbulb\ };
        }
    ]
        Useful tip with a light bulb icon in the title.
    \end{tcolorbox}
    
    \begin{tcolorbox}[
        enhanced,
        colback=ugmLightGrey,
        colframe=ugmBlue,
        title=Question Box,
        fonttitle=\bfseries,
        overlay unbroken and first={
            \node[anchor=west, inner sep=0pt] 
            at ([xshift=10pt]frame.north west) 
            {\faQuestion\ };
        }
    ]
        Question or quiz with a question icon in the title.
    \end{tcolorbox}
\end{frame}

% ----- SECTION: REFERENCES AND BIBLIOGRAPHY -----
\section{References and Bibliography}

\begin{frame}
    \frametitle{Manual Bibliography}
    
    \begin{thebibliography}{9}
        \bibitem{ref1} 
        Author, A. (2023). 
        \textit{Title of the paper}. 
        Journal Name, 10(2), 123-145.
        
        \bibitem{ref2} 
        Author, B., \& Author, C. (2022). 
        \textit{Title of the book}.
        Publisher Name.
    \end{thebibliography}
\end{frame}

\begin{frame}
    \frametitle{Bibliography Tips}
    
    \begin{tcolorbox}[colback=ugmLightGrey,colframe=ugmBlue,title=Bibliography Options]
        \begin{itemize}
            \item \textbf{Manual bibliography:} Use \texttt{thebibliography} environment
            \item \textbf{BibTeX:} Use \texttt{bibliographystyle} and \texttt{bibliography} commands
            \item \textbf{BibLaTeX:} More modern approach with \texttt{printbibliography}
        \end{itemize}
    \end{tcolorbox}
    
%     \begin{tcolorbox}[colback=yellow!10,colframe=yellow!50!black,title=\faLightbulb\ Tip]
%         For a BibTeX-based approach:
        
%         \begin{verbatim}
% \bibliographystyle{plain}
% \bibliography{references}
%         \end{verbatim}

%         Where \texttt{references.bib} is your BibTeX database file.
%     \end{tcolorbox}
\end{frame}

% ----- CONCLUSION -----
\section{Conclusion}

\begin{frame}
    \frametitle{Additional Resources}
    
    \begin{columns}[T]
        \begin{column}{0.48\textwidth}
            \begin{tcolorbox}[colback=ugmLightGrey,colframe=ugmBlue,title=LaTeX Resources]
                \begin{itemize}
                    \item Overleaf Documentation
                    \item LaTeX Wikibook
                    \item TeX Stack Exchange
                    \item Beamer User Guide
                    \item PGFPlots Manual
                    \item TikZ Documentation
                \end{itemize}
            \end{tcolorbox}
        \end{column}
        
        \begin{column}{0.48\textwidth}
            \begin{tcolorbox}[colback=ugmLightGrey,colframe=ugmBlue,title=Contacts and Collaboration]
                \begin{itemize}
                    \item Email: harunsixsixfour@gmail.com
                    \item Repository: \href{https://github.com/runsdev/ugm-snowyblue-beamer-template}{ugm-snowyblue-beamer-template}
                    \item Last updated: \today
                \end{itemize}
            \end{tcolorbox}
        \end{column}
    \end{columns}
\end{frame}

\end{document}